\renewcommand{\section}[1]{\paragraph{#1}}
\renewcommand{\subsection}[1]{\paragraph{#1}}
\renewcommand{\subsubsection}[1]{\paragraph{#1}}
\renewcommand{\paragraph}[1]{\textbf{#1}\quad}
\hypertarget{ki-einsatz-in-wissenschaftlichen-arbeiten}{%
\section{KI-Einsatz in wissenschaftlichen
Arbeiten}\label{ki-einsatz-in-wissenschaftlichen-arbeiten}}

\textbf{Created:} 7.12.2025, 09:30:22 \textbf{Updated:} 7.12.2025,
15:14:27 \textbf{Exported:} 7.12.2025, 15:14:49 \textbf{Model:}
claude-sonnet-4-5-20250929 \textbf{Link:}
\url{https://claude.ai/chat/2de0e0bb-3080-43cb-914b-0af284fd2e90}

\begin{center}\rule{0.5\linewidth}{0.5pt}\end{center}

\hypertarget{user}{%
\subsection{User}\label{user}}

\textbf{2025-12-07T08:30:24.146Z}

du bist ein Beispiel Prompt für KI in Wissenschaftlichen arbeiten.
Erkläre mir was ich beim Einsatz von KI in Wissenschaftlichen arbeiten
beachten muss

\hypertarget{claude}{%
\subsection{Claude}\label{claude}}

\textbf{2025-12-07T08:30:49.593Z}

\hypertarget{leitfaden-fuxfcr-den-einsatz-von-ki-in-wissenschaftlichen-arbeiten}{%
\section{Leitfaden für den Einsatz von KI in wissenschaftlichen
Arbeiten}\label{leitfaden-fuxfcr-den-einsatz-von-ki-in-wissenschaftlichen-arbeiten}}

Als KI-Assistent möchte ich dir einen umfassenden Überblick geben, was
du beim Einsatz von KI-Tools in wissenschaftlichen Arbeiten beachten
solltest.

\hypertarget{transparenz-und-dokumentation}{%
\subsection{Transparenz und
Dokumentation}\label{transparenz-und-dokumentation}}

Der wichtigste Grundsatz ist absolute Transparenz. Du musst jeden
Einsatz von KI in deiner Arbeit offenlegen. Das bedeutet konkret, dass
du dokumentieren solltest, welche KI-Tools du wann, wofür und in welchem
Umfang eingesetzt hast. Viele Universitäten verlangen mittlerweile eine
KI-Nutzungserklärung, ähnlich der Eigenständigkeitserklärung.

\hypertarget{verantwortung-und-geistiges-eigentum}{%
\subsection{Verantwortung und geistiges
Eigentum}\label{verantwortung-und-geistiges-eigentum}}

Du bleibst vollständig verantwortlich für alle Inhalte deiner Arbeit,
auch wenn sie mit KI-Unterstützung entstanden sind. Das bedeutet, dass
du jeden von KI generierten Inhalt kritisch prüfen, verstehen und
gegebenenfalls anpassen musst. Du kannst dich nicht darauf berufen, dass
``die KI einen Fehler gemacht hat'' -- die akademische Verantwortung
liegt bei dir.

\hypertarget{zitier--und-quellenregeln}{%
\subsection{Zitier- und Quellenregeln}\label{zitier--und-quellenregeln}}

KI-generierte Inhalte sollten als solche gekennzeichnet werden. Die
genaue Zitierweise variiert je nach Fachbereich und Institution, aber
generell gilt: Wenn KI substantiell zur Erstellung eines Textabschnitts
beigetragen hat, muss dies kenntlich gemacht werden. Beachte, dass
KI-Outputs selbst keine zitierfähigen Quellen sind -- du musst
Originalquellen recherchieren und überprüfen.

\hypertarget{erlaubte-und-problematische-einsatzbereiche}{%
\subsection{Erlaubte und problematische
Einsatzbereiche}\label{erlaubte-und-problematische-einsatzbereiche}}

KI kann sinnvoll unterstützen bei: Brainstorming und Ideenentwicklung,
Strukturierung von Gedanken, Überarbeitung von Formulierungen,
Übersetzungshilfen, Erklärung komplexer Konzepte,
Literaturrecherche-Unterstützung und technischer Unterstützung bei
Programmierung oder Datenanalyse.

Kritisch oder unzulässig ist KI-Einsatz für: das Schreiben kompletter
Textabschnitte ohne eigene Überarbeitung, die Erstellung von Analysen
ohne eigenes Verständnis, das Generieren von Quellenangaben (KI
halluziniert häufig falsche Quellen), originäre Forschungsleistungen,
die du selbst erbringen sollst, und methodische Entscheidungen ohne dein
eigenes Verständnis.

\hypertarget{qualituxe4tssicherung}{%
\subsection{Qualitätssicherung}\label{qualituxe4tssicherung}}

KI macht Fehler -- und das häufig. Du musst alle Fakten, Daten und
Quellenangaben, die eine KI liefert, unabhängig überprüfen. KI neigt zu
``Halluzinationen'', also dem Erfinden plausibler klingender, aber
falscher Informationen. Gerade bei Literaturangaben ist extreme Vorsicht
geboten.

\hypertarget{pruxfcfungsrechtliche-aspekte}{%
\subsection{Prüfungsrechtliche
Aspekte}\label{pruxfcfungsrechtliche-aspekte}}

Informiere dich über die spezifischen Regelungen deiner Hochschule und
deines Fachbereichs. Diese können sehr unterschiedlich sein -- von
völligem Verbot bis zu expliziter Erlaubnis mit Dokumentationspflicht.
Bei Verstößen gegen die Regeln drohen ernsthafte Konsequenzen bis zur
Bewertung als Täuschungsversuch.

\hypertarget{datenschutz-und-vertraulichkeit}{%
\subsection{Datenschutz und
Vertraulichkeit}\label{datenschutz-und-vertraulichkeit}}

Achte darauf, keine sensiblen oder vertraulichen Daten in KI-Tools
einzugeben, besonders wenn du mit nicht-öffentlichen Forschungsdaten
arbeitest. Viele KI-Dienste speichern Eingaben für Trainingszwecke.

\hypertarget{wissenschaftliche-integrituxe4t}{%
\subsection{Wissenschaftliche
Integrität}\label{wissenschaftliche-integrituxe4t}}

Der Einsatz von KI sollte deine wissenschaftliche Kompetenzentwicklung
unterstützen, nicht ersetzen. Das Ziel einer wissenschaftlichen Arbeit
ist nicht nur das Endprodukt, sondern auch der Lernprozess. Überlege
kritisch, wo KI-Unterstützung sinnvoll ist und wo sie dein eigenes
Lernen behindern würde.

Mein Rat: Nutze KI als intelligentes Werkzeug, das dich unterstützt,
aber behalte immer die Kontrolle und Verantwortung für deine
wissenschaftliche Arbeit. Im Zweifelsfall sprich mit deinen Betreuern
über geplante KI-Nutzung.

\hypertarget{user-1}{%
\subsection{User}\label{user-1}}

\textbf{2025-12-07T14:12:10.771Z}

Danke, erstelle noch mal inhalte damit ich mehrere Seiten als PDF Habe

\hypertarget{claude-1}{%
\subsection{Claude}\label{claude-1}}

\textbf{2025-12-07T14:14:27.519Z}

Ich habe dir einen umfassenden Leitfaden erstellt, der mehrere Seiten im
PDF-Format umfasst. Der Leitfaden enthält:

\begin{enumerate}
\def\labelenumi{\arabic{enumi}.}
\tightlist
\item
  \textbf{Grundprinzipien} - Transparenz, Eigenverantwortung, KI als
  Werkzeug
\item
  \textbf{Dokumentation und Offenlegung} - Was und wie dokumentiert
  werden muss
\item
  \textbf{Zitieren und Quellenangaben} - Korrekte Zitierweisen und
  Fallstricke
\item
  \textbf{Erlaubte Anwendungsbereiche} - Vom Brainstorming bis zur
  technischen Unterstützung
\item
  \textbf{Problematische Bereiche} - Was vermieden werden sollte
\item
  \textbf{Qualitätssicherung} - Typische Fehler und wie man sie
  vermeidet
\item
  \textbf{Institutionelle Regelungen} - Hochschul- und fachspezifische
  Vorgaben
\item
  \textbf{Datenschutz} - Umgang mit sensiblen Daten
\item
  \textbf{Kompetenzentwicklung} - KI als Lernhilfe richtig einsetzen
\item
  \textbf{Praktische Empfehlungen} - Konkrete Tipps für den
  Arbeitsablauf
\item
  \textbf{Zukunftsperspektiven} - Wie sich Standards entwickeln
\item
  \textbf{Checkliste} - Zur praktischen Anwendung
\end{enumerate}

Du kannst das Dokument jetzt aus dem Artifact kopieren und in einem
Texteditor oder direkt als Markdown-Datei speichern. Für ein PDF kannst
du es dann über einen Markdown-zu-PDF-Konverter oder über ``Drucken →
Als PDF speichern'' in deinem Browser umwandeln.

Brauchst du noch Anpassungen oder zusätzliche Abschnitte?
