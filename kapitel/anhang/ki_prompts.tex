\newpage
\section*{Anhang A: Dokumentation der KI-Prompts}
\addcontentsline{toc}{section}{Anhang A: Dokumentation der KI-Prompts}

Im Rahmen dieser Arbeit wurden KI-Tools zur Unterstützung bei verschiedenen Aufgaben eingesetzt. Gemäß den Richtlinien der FOM Hochschule werden hier alle verwendeten Prompts transparent dokumentiert.

\subsection*{A.1 KI-Assistent - Tabellenerstellung}
\textbf{Verwendungszweck:} Erstellung einer strukturierten Tabelle zur Übersicht der Forschungsmethodik

\textbf{Prompt:}
\begin{quote}
\textit{„Erstelle eine Übersichtstabelle für die Forschungsmethodik in LaTeX.
 Die Tabelle soll alle relevanten Kategorien zeigen und die für diese Arbeit gewählten Ansätze hervorheben.
 Stelle sicher, dass die Tabelle gut lesbar formatiert ist und alle notwendigen Informationen enthält."}

\textit{Gewählte Einordnung für diese Arbeit:}
\textit{- Forschungsansatz: Qualitativ-quantitativ}
\textit{- Datenerhebung: Systematische Literaturrecherche}
\textit{- Analysemethoden: Inhaltanalyse und Statistik}
\textit{- Zielgruppe: Akademische Forscher und Praktiker}
\textit{- Zeitrahmen: Repräsentative Quellen der letzten Jahre}}
\end{quote}

\subsection*{A.2 KI-Assistent - Concept Map Erstellung}
\textbf{Verwendungszweck:} Systematische Entwicklung einer Concept Map mit Suchbegriffen für die Literaturrecherche

\textbf{Prompt:}
\begin{quote}
\textit{„Erstelle eine umfassende Concept Map für das Forschungsthema.}

\textit{Die Concept Map soll:}
\textit{- Alle relevanten Suchbegriffe und Konzepte für eine systematische Literaturrecherche enthalten}
\textit{- In thematische Kategorien strukturiert sein}
\textit{- Sowohl deutsche als auch englische Fachbegriffe berücksichtigen}
\textit{- Praxisrelevante Technologien und Methoden einschließen}
\textit{- Eine Grundlage für Boolean-Suchstrategien in wissenschaftlichen Datenbanken bieten}
\textit{Anschließend formatiere die Ergebnisse als LaTeX-Tabelle"}
\end{quote}

\subsection*{A.3 KI-Assistent - Literaturrecherche und -filterung}
\textbf{Verwendungszweck:} KI-gestützte Analyse und Filterung der Literaturquellen zur Identifikation der relevantesten Publikationen für die Forschungsfrage

\textbf{Prompt:}
\begin{quote}
\textit{„Analysiere die BibTeX-Datei mit den Literaturrecherche-Ergebnissen und filtere nur die relevantesten Quellen für das Forschungsthema.}

\textit{Bitte:}
\textit{1. Lies die gesamte BibTeX-Datei}
\textit{2. Analysiere jede Quelle basierend auf Titel und Abstract hinsichtlich der Relevanz}
\textit{3. Behalte nur Einträge, die DIREKT relevant sind für das Forschungsthema}
\textit{4. SCHLIESSE irrelevante Quellen aus}
\textit{5. Erstelle eine neue BibTeX-Datei mit nur den relevanten Einträgen}
\textit{6. Gib eine Zusammenfassung mit Statistiken und identifizierten Kategorien"}
\end{quote}

\textbf{Ergebnis der KI-Analyse:}
\begin{itemize}
    \item Anzahl der analysierten Quellen
    \item Anzahl der selektierten hochrelevanten Quellen
    \item Identifizierte Hauptkategorien
    \item Ausgabedatei mit gefilterten Quellen
\end{itemize}

\subsection*{A.4 KI-Assistent - Automatisierungsskript}
\textbf{Verwendungszweck:} Entwicklung eines Skripts zur automatisierten Datenverarbeitung und Analyse

\textbf{Prompt:}
\begin{quote}
\textit{„Erstelle ein Skript zur automatisierten Verarbeitung von Forschungsdaten.}

\textit{Das Skript soll:}
\textit{1. Dateien aus einem Verzeichnis einlesen und verarbeiten}
\textit{2. Daten extrahieren und analysieren}
\textit{3. Ergebnisse mit bestehenden Daten abgleichen}
\textit{4. Häufigkeiten und Muster identifizieren}
\textit{5. Reports in verschiedenen Formaten generieren}

\textit{Technische Anforderungen:}
\textit{- Robuste Fehlerbehandlung}
\textit{- Unterstützung verschiedener Dateiformate}
\textit{- Logging für Nachvollziehbarkeit}
\textit{- Export der Ergebnisse in strukturierten Formaten"}
\end{quote}

\textbf{Ergebnis der Skript-Anwendung:}
\begin{itemize}
    \item Anzahl der verarbeiteten Dateien
    \item Anzahl der extrahierten Datenelemente
    \item Identifizierte Muster und Häufigkeiten
    \item Generierte Ausgabedateien
\end{itemize}

\subsection*{A.5 KI-Assistent - Textüberarbeitung und Formatierung}
\textbf{Verwendungszweck:} Unterstützung bei der Textüberarbeitung und LaTeX-Formatierung

\textbf{Prompt:}
\begin{quote}
\textit{„Überarbeite den folgenden Text und formatiere ihn korrekt in LaTeX.}

\textit{Bitte:}
\textit{1. Korrigiere Grammatik und Rechtschreibung}
\textit{2. Verbessere die Satzstruktur und Lesbarkeit}
\textit{3. Formatiere Zitate und Verweise korrekt}
\textit{4. Stelle sicher, dass alle LaTeX-Befehle korrekt verwendet werden}
\textit{5. Prüfe die Konsistenz der Formatierung"}
\end{quote}