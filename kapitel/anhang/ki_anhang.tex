\section{Dokumentation der KI-Prompts}

Im Rahmen dieser Arbeit wurden KI-Tools zur Unterstützung bei verschiedenen Aufgaben eingesetzt. Gemäß den Richtlinien der FOM Hochschule werden hier alle verwendeten Prompts transparent dokumentiert.

\subsection{Beispiel ChaGPT mit Screenshot}
\textbf{Verwendungszweck:} Beispiel des Einsatzes von ChatGPT mit Screenshots.

\begin{figure}[H]
    \centering
    \caption[]{Beispiel des Einsatzes von ChatGPT mit Screenshots}
	\label{fig:Beispiel ChaGPT mit Screenshot}
    \includegraphics[width=1\textwidth]{abbildungen/ki_belege/ChatGPT_beispiel_2025-12-07 10-28-58}
\end{figure}


\newpage
\subsection{Beispiel Claude mit Markdown Export zu PDF}\label{KI1}
\textbf{Verwendungszweck:} Beispiel des Einsatzes von Claude mit Markdown-Export zu PDF.

\textbf{KI-Modell:} Claude Sonnet 4.5 (Anthropic) \\
\textbf{Datum:} 7. Dezember 2025 \\
\textbf{Thema:} Beispiel Einsatz KI Einsatz in Wisscenschaftlichen Arbeiten \\
\textbf{Literatur:} \kicite{KI_Claude_2025_Beispiel}{Claude Sonnet 4.5}{2025}

%ACHTUNG Die Scale und Offset einstellungen sind Feinustiert auf das aktuelle Format. Wenn ihr was ändert oder es "shief aussieht bitte in der thesis_main.tex die Zeilen "\usepackage{showframe}" & "\usepackage[showboxes]{textpos}" suchen und auskommentieren. Damit werden dann die "boxen" um die Elemente angezeigt und ihr könnt das noch mal fine tunen

\newcounter{includepdfpage}
\includepdf[
    pages=-,
    pagecommand={\stepcounter{includepdfpage}\ref{KI1} Chatverlauf Seite \theincludepdfpage\thispagestyle{fancy}},
    frame=true,
    scale=0.715,
    offset=10mm 11.5mm
    landscape=false
]{abbildungen/ki_belege/claude_prompt_als_pdf_exportieren.pdf}


