\newpage
\pagenumbering{gobble} % Keine Seitenzahlen

%-----------------------------------
% KI-Transparenzerklärung
%-----------------------------------
\section*{KI-Disclaimer}

\subsection*{Erklärung zum Einsatz digitaler Unterstützung in der wissenschaftlichen Arbeit:}

Im Rahmen der Erstellung dieser Arbeit wurden moderne digitale Hilfsmittel und KI-basierte Systeme genutzt, um den Forschungs- und Schreibprozess effizient zu unterstützen. Dabei wurden die aktuellen Vorgaben der FOM Hochschule für Oekonomie und Management (Aktueller Stand: März 2024), die DFG-Leitlinien zur Sicherung guter wissenschaftlicher Praxis (Kodex 2023) sowie die Datenschutzbestimmungen der DSGVO strikt eingehalten.

Die eingesetzten Technologien dienten der Optimierung von Arbeitsschritten, ohne die Eigenständigkeit oder Originalität der Arbeit zu beeinträchtigen. Konkret fanden sie Anwendung in den folgenden Bereichen:

\begin{itemize}
    \item \textbf{Textoptimierung:} Verbesserung von Formulierungen und der Textstruktur durch \textbf{Claude Code}.
    \item \textbf{Textstrukturierung:} Anpassungen der Struktur der Arbeit mittels \textbf{Claude Code}
    \item \textbf{Entwicklung von Inhalten:} Unterstützung bei der Entwicklung und Bewertung von Zielen, Anforderungen und Projektstrukturierung durch \textbf{Claude Code}\ und \textbf{ChatGPT}.
    \item \textbf{Literaturarbeit:} Unterstützung bei der Recherche und Analyse von Fachliteratur und Produkt Dokumentationen mithilfe von \textbf{Claude Code} und \textbf{Perplexity}.
    \item \textbf{Qualitätsprüfung:} Überprüfung auf sprachliche und formale Korrektheit durch \textbf{Claude Code}.
\end{itemize}

Alle wesentlichen inhaltlichen Entscheidungen sowie die kritische Prüfung und Interpretation wurden ausschlie\ss lich von mir als Verfasser der Arbeit vorgenommen. Sämtliche Quellen wurden nach wissenschaftlichen Standards geprüft und korrekt zitiert. Von digitalen Hilfsmitteln und KI-generierten Inhalten habe ich nur solche übernommen, die ich persönlich validiert und bei Bedarf überarbeitet habe.

\subsection*{Grundsätze bei der Nutzung von KI-Systemen:}

Bei der Verwendung von KI-Systemen wurden folgende Prinzipien berücksichtigt:

\begin{itemize}
    \item Keine Eingabe oder Verarbeitung personenbezogener oder vertraulicher Daten
    \item Strikte Einhaltung der Datenschutzrichtlinien der genutzten Tools
    \item Eingrenzung der KI-Nutzung auf rein unterstützende Funktionen
    \item Kritische Überprüfung und Anpassung generierter Inhalte
    \item Vollständige Dokumentation der KI-Nutzung
\end{itemize}

\subsection*{Eingesetzte Tools und Anwendungen:}

Es wurden die folgenden digitalen Werkzeuge genutzt:

\begin{table}[H]
\centering
\renewcommand{\arraystretch}{1.3}
\begin{tabularx}{\textwidth}{|l|X|l|}
\hline
\rowcolor{gray!30}
\textbf{Kategorie} & \textbf{Beschreibung} & \textbf{Quelle} \\
\hline

% --- Claude Code ---
\multicolumn{3}{|l|}{\cellcolor{gray!15}\textbf{Claude Code}} \\
\hline
Textstrukturierung
    & Einbindung eines KI-Disclaimers
    & \kicite{Claude_KI_Disclaimer}{Claude Code}{2026} \\
    & Analyse und Umsetzung zu Formatierung der KI-Nutzungsübersicht als Tabelle
    & \kicite{Claude_KI_overview_table}{Claude Code}{2026} \\
    & Einbinden der Zielabhängigkeiten-Matrix ins Dokument
    & \kicite{Claude_goal_dependency_matrix}{Claude Code}{2026} \\
    & Einbinden der Prioritäts-Komplexitäts-Matrix inkl. Formulierung
    & \kicite{Claude_Einbinden_Der_Prioritätskomplexitätsmatrix_Inkl_Formulierung}{Claude Code}{2026} \\
    & Erstellung des Gantt-Diagramms mit integrierten Quality Gates
    & \kicite{Claude_Finds_Projektplan_Ganttdiagramm_Mit_Integrierten_Quality}{Claude Code}{2026} \\
\hline
Textoptimierung
    & Verbesserung von Formulierungen und Textstruktur
    & -- \\
    & Formulierungshilfe zum hybriden Ansatz des Vorgehensmodells
    & \kicite{Claude_Formulierungshilfe_Zu_Kombination_Und_Hybriden_Ansatz}{Claude Code}{2026} \\
    & Formulierungshilfe zum FINDS-Aufbau auf dem SPM-Modul
    & \kicite{Claude_Formulierungshilfe_Finds_Aufbau_Auf_Spm_Modul}{Claude Code}{2026} \\
    & Formulierungshilfe für die Implementierungsplan-Einleitung
    & \kicite{Claude_Formulierungshilfe_Für_Implementierungsplaneinleitung}{Claude Code}{2026} \\
\hline
Inhaltsentwicklung
    & Erarbeitung funktionaler Anforderungen
    & \kicite{Claude_Anforderungsliste_Funktionale_Anforderungen}{Claude Code}{2026} \\
    & Erarbeitung nicht-funktionaler Anforderungen
    & \kicite{Claude_Anforderungsliste_Nichtfunktionale_Anforderungen}{Claude Code}{2026} \\
    & Scope-Anpassung und Traceability-Matrix
    & \kicite{Claude_Anforderungsliste_Scope_Anpassung_Z6}{Claude Code}{2026} \\
    & Erläuterung der Scope-Reduktion im Success Pack
    & \kicite{Claude_Erläuterung_Der_Scope_Reduktion_Im_Success}{Claude Code}{2026} \\
    & Detaillierte Erläuterung der Phasen und Aufgaben
    & \kicite{Claude_Detailierte_Erläuterung_Phasen_Und_Aufgaben}{Claude Code}{2026} \\
    & Ausformulierung Projektressourcen und Kosten
    & \kicite{Claude_Ausformulierung_Zu_Projektressourcen_Und_Kosten}{Claude Code}{2026} \\
    & Ausformulierung Projektdurchführung inkl. Soll-Ist-Vergleich
    & \kicite{Claude_Ausformulierung_Projektdurchfuehrung}{Claude Code}{2026} \\
    & Ausformulierung Lessons Learned
    & \kicite{Claude_Lessons_Learned}{Claude Code}{2026} \\
\hline
Literaturarbeit
    & Recherche zu ServiceNow Success Packs
    & \kicite{Claude_Success_Packs_Recherche}{Claude Code}{2026} \\
\hline

% --- Claude ---
\multicolumn{3}{|l|}{\cellcolor{gray!15}\textbf{Claude}} \\
\hline
Inhaltsentwicklung
    & Unterstützung Ausformulierung von Zielen und Nichtzielen sowie der Strukturierung dieser
    & \kicite{Claude_Goal_formulation}{Claude}{2026} \\
    & Unterstützung bei der Definition des Zielrahmens mit Timeboxing
    & \kicite{Claude_Zielrahmen_Mit_Timeboxing_Für_Projektarbeit_Ergänzen}{Claude}{2026} \\
\hline

\end{tabularx}
\end{table}

Die Verwendung dieser Tools wurde lückenlos dokumentiert. Diese Dokumentation umfasst:

\begin{itemize}
    \item Vollständige Protokolle der KI-Interaktionen als Chatverlauf mit PDF
    \item von der KI Erstellte Artefakte (z.B. Quellcode, Zusammenfassungen, etc.)
\end{itemize}

Für die Erstellung der Chat-Protokolle als PDF wurde ein einheitlicher, reproduzierbarer Prozess eingehalten: \\
Zunächst wurden die Chatverläufe aus den jeweiligen KI-Tools exportiert -- bei Claude Code im JSONL-Format und bei der Claude-Weboberfläche im JSON-Format. Anschlie\ss end wurden die exportierten Daten mithilfe eines eigens entwickelten Python-Skripts (\texttt{convert\_session.py}) in Markdown überführt und über Pandoc mit LuaLaTeX als PDF-Engine in einheitlich formatierte PDF-Dokumente konvertiert. Die resultierenden Dateien wurden im Projektverzeichnis unter \texttt{literatur/fluechtige\_quellen/ki\_chats/} abgelegt und in der Bibliografie mit dem Schlüsselwort \texttt{kichat} referenziert, sodass sie automatisch im KI-Verzeichnis dieser Arbeit erscheinen.

\begin{figure}[H]
\centering
\begin{tikzpicture}[
    node distance=0.6cm,
    box/.style={
        rectangle,
        draw=black!70,
        fill=gray!10,
        thick,
        text width=11cm,
        minimum height=1cm,
        align=center,
        font=\small
    },
    arrow/.style={
        -stealth,
        thick,
        gray!70
    }
]
    % Nodes
    \node[box] (chat) {
        \textbf{1. KI-Chat durchführen}\\[2pt]
        Interaktion mit Claude Code, Claude Web o.\,ä.
    };
    \node[box, below=of chat] (export) {
        \textbf{2. Chatverlauf exportieren}\\[2pt]
        Claude Code: JSONL-Format \quad|\quad Claude Web: JSON-Format
    };
    \node[box, below=of export] (convert) {
        \textbf{3. Konvertierung mit \texttt{convert\_session.py}}\\[2pt]
        Parsen der Rohdaten und Erzeugung einer strukturierten Markdown-Datei
    };
    \node[box, below=of convert] (pdf) {
        \textbf{4. PDF-Erzeugung via Pandoc + LuaLaTeX}\\[2pt]
        Markdown wird mit einheitlichem Layout in PDF konvertiert
    };
    \node[box, below=of pdf] (ablage) {
        \textbf{5. Ablage und Referenzierung}\\[2pt]
        PDF in \texttt{literatur/fluechtige\_quellen/ki\_chats/} ablegen,\\
        Eintrag in \texttt{literatur.bib} mit \texttt{keyword=\{kichat\}}
    };

    % Arrows
    \draw[arrow] (chat) -- (export);
    \draw[arrow] (export) -- (convert);
    \draw[arrow] (convert) -- (pdf);
    \draw[arrow] (pdf) -- (ablage);
\end{tikzpicture}
\caption*{Ablauf der KI-Chat-Protokoll-Erstellung}
\end{figure}

Die vollständigen Chat-Protokolle sind im angehängten elektronischen Repository zu dieser Arbeit einsehbar. Dabei wurde gewährleistet, dass keine sensiblen Daten oder urheberrechtlich geschützten Inhalte ohne Zustimmung verwendet wurden.

\subsection*{Transparenz und wissenschaftliche Integrität:}

Alle Abweichungen vom hier beschriebenen Einsatz digitaler Hilfsmittel sind separat kenntlich gemacht und entsprechend den Leitfäden der FOM (siehe Online-Campus oder Moodle) dokumentiert. Zudem wurden die spezifischen Vorgaben der \textbf{Fakultät Wirtschaftsinformatik} gemä\ss dem aktuellen \enquote{Leitfaden für wissenschaftliche Arbeiten, IT-Management \& Ingenieurwesen (Stand 03/24)}, berücksichtigt.

Mit dieser Erklärung möchte ich grö\ss tmögliche Transparenz über den Einsatz moderner Technologien schaffen und zugleich sicherstellen, dass alle akademischen Standards sowie geltendes Recht eingehalten wurden. Ich versichere, dass die vorliegende Arbeit trotz der digitalen Unterstützung meine eigenständige wissenschaftliche Leistung darstellt und alle Anforderungen an eine \textbf{Projektarbeit} erfüllt.

\par\medskip

\textbf{\myAutor}, \textbf{\myMatrikelNr}, \textbf{\myAkademischerGrad \myStudiengang}
